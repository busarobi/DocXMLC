\documentclass{article}

\usepackage{fullpage,amssymb,amsthm,amsmath}
\usepackage{natbib}
\usepackage{multirow}
\usepackage{array,graphicx}

\newtheorem{theorem}{Theorem}
\newtheorem{lemma}{Lemma}
\newtheorem{proposition}{Proposition}
\newtheorem{corollary}{Corollary}
\newtheorem{remark}{Remark}

\usepackage{algpseudocode}
\usepackage{algorithm}


\newcommand{\Gm}{L_{\max}}
\newcommand{\R}{\mathbb{R}}
\DeclareMathOperator*{\argmin}{argmin}
\DeclareMathOperator*{\argmax}{argmax}
\DeclareMathOperator*{\E}{\mathbb{E}}

\newcommand{\cD}{\mathcal{D}}
\newcommand{\cH}{\mathcal{H}}
\newcommand{\cL}{\mathcal{L}}
\newcommand{\cT}{\mathcal{T}}
\newcommand{\path}{\text{Path}}
\newcommand{\Path}[1]{\mathrm{Path}(#1)}
\newcommand{\pa}[1]{\mathrm{pa}(#1)}
\newcommand{\leaves}{d}

\renewcommand{\vec}[1]{\mathbf{#1}}
\newcommand{\bx}{\mathbf{x}}
\newcommand{\by}{\vec{y}}
\newcommand{\bY}{\vec{Y}}
\newcommand{\prob}{\mathbf{P}} 

\newcommand{\given}{\, | \,}

\newcommand{\Algo}[1]{\textsc{#1}}

\begin{document}

\section{Probabilistic Label Tree (PLT) model}

The data $\cD = \{ (\bx_{i},\by_{i})\}_{i=1,\dots,n}$ consists of feature vector $\bx_i \in \R^d$ and label vector $\by_i\in \{ 0,1\}^m$ tuples. The labels can be written in a matrix form $\bY = [y_{i,j}]$ whose $j$\/th column is denoted by $\bar{\by}_{j}$.

We will work with the set of rooted trees with $m$ leaves which is denoted by $\cT$. For sake of simplicity, we assume that the arity of the trees is $k=2$. The leaves of a tree $T$ is denoted by $\ell_1, \dots , \ell_m$. The path from the root to the leaf $\ell_i$ is denoted by $\path (i )$. 

%There is given a set of labeling functions $\Lambda$ which contains binary functions in the form of $\lambda: V(T) \mapsto \{0,1\}$. For a labeling function $\lambda$ that is compatible with $(\bx,\by)$, it holds that $\lambda(\ell_j) = y_{j}$ for all $j\in [m]$. The set of labeling functions that is compatible with $\bx$ is denoted by $\Lambda_{\bx}$. We shall concisely write 
%$\lambda_{\bx_i}=\lambda_i$ and 
%$\Lambda_{\bx_i} = \Lambda_{i}$. In words, $\Lambda_{\bx}$ contains every binary assignment to the inner nodes which matches with the label vector $\by$ of $\bx$ on the leaves.

We are also given a function class $\cH : \R^d \mapsto [0,1]$ which set contains probabilistic classifiers of our choice, for example, set of logistic regressors. We assign a classifier from $\cH$ to each node of the tree $T$. We shall index this (possibly multi-) subset of classifiers by the element of $V(T)$ as $H = \{ h_{v} \in \cH : v\in V(T) \}$. The set of all multi-set from $\cH$ with $k$ element is denoted by  $\cH_k$.

The Probabilistic Label Tree (PLT) model is defined by a tree $T \in \cT$ in which each inner node is assigned to a classifier from $\cH$ that is denoted by $H$. In addition, each label of the tree is assigned to a label $[m]$. The conditional probability estimate $\eta( \bx, j ) (\approx \prob ( y_j \vert \bx))$ for an instance $\bx$ with respect to the \Algo{PLT} model with tree $T$ and classifier set $H$, is given as
\[
\eta( \bx, j ) = \prod_{v \in \path(\ell_j)} h_v (\bx) 
\]
Note that this expression depends on the tree structure via the path from the root to the leaf $\ell_j$. If this does not lead any confusion, we do not indicate that $\eta$ depends on $T,H$, that is, $\eta_{T,H}( \bx_i, j ) = \eta( \bx_i, j ) $.

The likelihood of $\cD$ can be written in the form of
\begin{align}
\cL ( T, H ) = \prod_{i=1}^{n} \prod_{j=1}^m \eta_{T,H,} ( \bx_i , j )^{y_{i,j}} (1-\eta_{T,H} ( \bx_i , j ))^{(1-y_{i,j})}
\label{eq:mllikelihood_labelkindependence}
\end{align}
when we assume independence between the labels. There are several multi-label likelihood function which aims at capturing the dependence of labels~\citep{ZhangS12}. 


\section{Tree structure}

\subsection{Constant learner}


Assume a set of constant classifier $\cH_{C} = \{h_p\in \cH : p\in [0,1] \}$ where $h_p(\bx) = p$ for all $\bx \in \R^d$. Using $\cH_{C}$, a constant value will assigned to each inner node $v$ of tree $T$ that is denoted by $p_v$. Accordingly, $\eta( \bx, j ) = \prod_{v \in \path(\ell_j)} p_v$. 

First, given a tree $T$, we shall find the optimal constant classifiers that maximizes (\ref{eq:mllikelihood_labelkindependence}). We shall decompose this problem and consider the restricted problem for a subtree $T_v$ of $T$ rooted in an inner node $v\in V(T)$. One can define the conditional estimate of $\bx$ based on a subtree $T_v$ if $v\in \path (\ell_j)$ as 
\[
\eta_{T_v, H} ( j) = \eta_{T_v, H} (\bx , j) = \prod_{v^{\prime} \in \path(\ell_j) \setminus \path(v)}  p_{v^{\prime}}
\]
Let us define the likelihood function restricted to $T_v$ as
\begin{align}
\cL ( T_v, H ) = \prod_{i=1}^{n} \prod_{j : \ell_j \in V(T_v) } \eta_{T_v,H,} ( \bx_i , j )^{y_{i,j}} (1-\eta_{T_v,H} ( \bx_i , j ))^{(1-y_{i,j})}
\label{eq:mllikelihood_restricted}
\end{align}

The conditional estimate that maximizes $\cL ( T_v, H )$ for a given subtree $T_v$ is denoted by $\eta_{T_v}^* (\bx , j)$. Assume that we know that optimal conditional estimates for two sibling nodes $v'$ and $v''$, and try to find out what is the optimal constant classifier for the parent $v$ of $v'$ and $v''$. We start by analyzing a more simple case when the sibling node of a leaf can be only leaf and the sibling of an inner node can be only inner node.

\begin{itemize}
    \item Inner node $v\in V(T)$ with two leaves $\ell_j$ and $\ell_k$ as children: in this case the likelihood boils down to
    \[
    \cL ( T_v, H ) = \prod_{i=1}^{n} p_v^{y_{i,j}} (1-p_v)^{(1-y_{i,j})}p_v^{y_{i,k}} (1-p_v)^{(1-y_{i,k})} = p_v^{s_j+s_k} (1-p_v)^{2n-(s_j+s_k)}
    \]
    where $s_j = \# \{ i \in [n] : y_{i,j}=1\}$. This coincides with the likelihood of a binomial with $2n$ trial and $s_i+s_j$ success, this the optimal $p_v=\tfrac{s_i+s_j}{2n}$.
    \item Inner node $v\in V(T)$ with two inner node $v'$ and $v''$ as children: in this case let us assume that we know the value of $\eta_{T_{v'}}^* (\bx , j)$ and $\eta_{T_{v''}}^* (\bx , j)$ for all $j\in [m]$.
\end{itemize}


%Furthermore, we define the Hamming distance $d_H( \bar{y}_i, \bar{y}_j) = \# \{ \ell \in [n] : y_{i,\ell} \neq y_{j,\ell} \}$ that is denoted by $d_{i,j}$.





\section{Training}

The naive way of optimizing \ref{eq:mllikelihood_labelkindependence} is computationally expensive, because one would need to compute the gradient for all path, and update the inner nodes based on the cumulative gradients. 

\subsection{Left, Right, Left-Right view}

Each internal node $t$ of a tree has three associated random variables: $z_{l(t)}$, $z_{r(t)}$, and $z_{lr(t)}$. The first one, $z_{l(t)}$, says that $\bx$ has positive label(s) in the left subtree only, the second one, $z_{r(t)}$, says that $\bx$ has positive label(s) in the right subtree, and the last one, $z_{rl(t)}$, says that there are positive labels in both subtrees. Let us assume without loss of generality that $\Path{\ell_j}$ consists of nodes being the left children of the parent nodes. Then, the marginal probability $\eta_{T,H,\lambda_i} ( \bx_i , j )$ can be defined by:
\begin{equation}
\eta(\bx, j) = \prod_{t \in \Path{\ell_j}}  \left ( \prob(z_{l(t)} = 1 \given z_{\pa{t}} =1, \bx) + \prob(z_{lr(t)} = 1 \given z_{\pa{t}} =1, \bx) \right ) \,.
\label{eqn:probabilistic_tree}
\end{equation}

Let us now define the maximum likelihood of the tree. The main problem is to express the probability of  $(1-\eta_{T,H,\lambda_i} ( \bx_i , j ))$, i.e., that the $j$-th label is not positive. By using the tree, we can write this in the following way:
\[
(1-\eta_{T,H,\lambda_i} ( \bx_i , j )) = \sum_{t \in \Path{\pa{\ell_j}}}   \prob(z_{r(t)} = 1 | \bx)  \prod_{t' \in \Path{\pa{t}}} \left ( \prob(z_{l(t')} = 1 | \bx) + \prob(z_{lr(t')} = 1 \given \bx) \right )  \,.
\]

TODO: Maximum likelihood problem, inference problem


\subsection{\Algo{PLT}}

$z_{l^{\prime}(t)} = z_{l(t)} \vee z_{lr(t)}$ 

\begin{equation}
\eta(\bx, j) = \prod_{t \in \Path{\ell_j}} \prob(z_{l^{'}(t)} = 1 \given z_{\pa{t}} =1, \bx)\notag 
\end{equation}


\[
(1-\eta_{T,H,\lambda_i} ( \bx_i , j )) = \sum_{t \in \Path{\pa{\ell_j}}}   \prob(z_{r(t)} = 1 | \bx)  \prod_{t' \in \Path{\pa{t}}} \prob(z_{l^{'}(t)} = 1 \given z_{\pa{t}} =1, \bx)  \notag
\]



% \subsection{Relation to Ancestral Maximum Likelihood}

% Assume a set of simple classifier $\cH_{C} = \{h_p\in \cH : p\in [0,1] \}$ where $h_p(\bx) = p$ for all $\bx \in \R^d$. Thus using s set of simple classifier, a constant value will assigned to $0$ that is denoted by $p_v$.

% Likelihood with simple classifiers can be written as
% \begin{align}
% \cL ( T, H, \{ \lambda_i, \dots, \lambda_n\} ) 
%   & = \prod_{i=1}^{n} \prod_{j=1}^m \left[ \prod_{v \in \path(\ell_j)} \left( \lambda_i (v) p_v + (1-\lambda_i (v) ) (1-p_v) \right)\right]^{y_{i,j}} \times \notag \\ & ~~~~~~~~~~~~~~~~~~~~ \left[ 1-\prod_{v \in \path(\ell_j)} \left( \lambda_i (v) p_v + (1-\lambda_i (v) ) (1-p_v) \right) \right]^{1-y_{i,j}}  \notag
% %   & = \prod_{j=1}^m \prod_{v \in \path(\ell_j)}\prod_{i\in I_i^+}  \left[ \lambda_i (v) p_v + (1-\lambda_i (v) ) (1-p_v) \right] \prod_{i\in I_i^-} \left[ 1-\left( \lambda_i (v) p_v + (1-\lambda_i (v) ) (1-p_v) \right) \right] \notag \\
% %   & = \prod_{j=1}^m \prod_{v \in \path(\ell_j)} p_{v}^{s_v} (1-p_v)^{n^+-s_v} \notag \\
% %   & = \prod_{v \in V(T)} p_{v}^{\leaves (v) s_v} (1-p_v)^{\leaves (v)(n-s_v)} \notag
% \end{align}
% where $s_v = \sum_{i=1}^n \lambda_i (v)$ and $\leaves (v)$ is the number of leaves below $v$ in the tree $T$, $\leaves( \ell_i ) = 1$ and $I_j^{+} = \{ i \in [n]: y_{i,j}=1 \}$ is index of positive labels of $(\bx_i, \by_i)$, and $I_i^{-}$ is the index set of negative labels. $n^+_i = \# I_i^{+}$

% The likelihood function bears a resemblance to the objective of Ancestral Maximum Likelihood (\Algo{AML})~\citep{Alon2010}, except the $d(v)$ in the exponent.

% TODO: Exact reduction to \Algo{AML}, then to show that inference is hard, i.e. find a label and labeling with fixed T and H that maximizes the likelihood

% \section{Positive leaf-to-root labeling}

% We restrict the set of labeling function in this section. We only consider such labeling function $\lambda$ that is compatible with $(\bx, \by )$ and assigns $1$ to each $v\in \path(\ell_i)$ whenever $y_i$, and $0$ otherwise. In this case, the labeling function is determined by the tree $T$ and labels uniquely. 

% The likelihood with $H$ and PLR labeling can be written as
% \begin{align}
% \cL ( T, H) 
%   & = \prod_{i=1}^{n} \prod_{j=1}^m \prod_{v \in \path(\ell_j)} \left( \lambda_i (v) p_v + (1-\lambda_i (v) ) (1-p_v) \right) \notag \\
%   & = \prod_{i=1}^{n} \prod_{j\in I_i^{+}}^m \prod_{v \in \path(\ell_j)} \left( \lambda_i (v) p_v + (1-\lambda_i (v) ) (1-p_v) \right) \notag \\
% \end{align}

\section{Sub-optimality of \Algo{LOM} criterion}

The \Algo{LOM} criterion is an evaluation criterion for PLTs with positive leaf-to-root labeling~\citep{Choromanska_Langford_2015}.

\bibliography{xmlc_references,treeb}
\bibliographystyle{icml2018}


\end{document}